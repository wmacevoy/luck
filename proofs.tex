\chapter*{Proofs}


\section{From the introduction}

\begin{theorem}{Range of Luck.} 
For any probability space,
\begin{equation}
0 \leq L(x) \leq 1 \,.
\end{equation}

Proof.  From the definition, $L(x)=|\Omega(x)|+\frac{1}{2} |\omega(x)|$, which is clearly non-negative, and $\Omega(x)$ and $\omega(x)$ are disjoint subsets of the probability space.  Here the $| \cdot |$ notation is the measure of these sets in the probability space, and their union is at most the whole space, so $L(x) \leq |\Omega(x) \cup \omega(x)| \leq 1$.
\end{theorem}

\begin{theorem}{Lucky values.}
If $L(x)$ is close to 1, then $p(x)$ is relatively small and most outcomes have a higher probability (you are lucky).

Proof.  Since $L(x)$ is close to 1, write $L(x)=1-\varepsilon(x)$, where $\varepsilon(x)$ is a small non-negative number.

First, since $\Omega(x)$ and $\omega(x)$ are disjoint subsets of the space of outcomes $X$, $|\Omega(x)|+|\omega(x)| \leq 1$, or 
\begin{equation*}
|\Omega(x)| \leq 1-|\omega(x)| \,.
\end{equation*}

Second, $L(x)=|\Omega(x)|+\frac{1}{2}|\omega(x)|$, which by the first inequality can be bounded as $1-\varepsilon(x)\leq  1-|\omega(x)|+\frac{1}{2}|\omega(x)|$, which can be rearranged as
\begin{equation*}
|\omega(x)| \leq 2 \varepsilon(x) \,.
\end{equation*}

Third, $|\Omega(x)|+\frac{1}{2}|\omega(x)| = 1-\varepsilon(x)$, or $|\Omega(x)|=1-\varepsilon(x)-\frac{1}{2}|\omega(x)|$, which by the second inequality,
\begin{equation}
|\Omega(x)| \geq 1-2 \varepsilon(x) \,.
\end{equation}
Thus at least a $1-2 \varepsilon(x)$ fraction of the probability space have a higher probability of occuring.
\end{theorem}

\begin{theorem}{Unlucky values.}
If $L(x)$ is close to 0, then $p(x)$ is comparatively large, and most outcomes would have a lower probability (you are unlucky).

Proof.  Since $L(x)$ is close to 1, write $L(x)=\varepsilon(x)$, where $\varepsilon(x)$ is a small non-negative number.

First, $L(x)=|\Omega(x)|+\frac{1}{2}|\omega(x)|$, so
\begin{equation*}
\omega(x) \leq 2\varepsilon(x) \,.
\end{equation*}

Second, $|\Omega(x)|+|\omega(x)|=|\Omega(x)|+\frac{1}{2}|\omega(x)|+\frac{1}{2}|\omega(x)| \leq 2\varepsilon(x)$, so
\begin{equation*}
|\Omega(x)|+|\omega(x)| \leq 2\varepsilon(x) \,.
\end{equation*}

Third, $|X-\Omega(x) \cup \omega(x)| \geq 1-|\Omega(x) \cup \omega(x)| \geq 1-2\varepsilon(x)$, or 
\begin{equation*}
|X-\Omega(x) \cup \omega(x)| \geq 1-2\varepsilon(x) \,.
\end{equation*}
Thus at least a $1-2 \varepsilon(x)$ fraction of the probability space have a lower probability of occuring.
\end{theorem}

\begin{theorem}{On average, luck is always 50:50.}
\begin{equation*}
E(L)=\frac{1}{2} \,.
\end{equation*}

Proof.  This is an application of the next theorem where $f(L)=L$.
\end{theorem}

\begin{theorem}{Smooth uniformity - finite space $X$.}
$E(f(L))=\int_0^1 f(L) dL+\varepsilon$, where $|\varepsilon| \leq \max|f''| \cdot \max |\omega|^2 / 24$.

Since the definition of luck only depends on the probabilities of outcomes, it is natural to consider a set of equivalence classes $[x]$ from $[X]$ defined by equal probabilities: $[x]=\{y \mid p(y)=p(x) \}$.  We also use the midpoint integration estimate:
\begin{equation*}
\int_a^b f(x) \, dx = (b-a)f(\frac{a+b}{2})+\frac{(b-a)^3}{24} f''(\xi), \text{where $a < \xi < b$.}
\end{equation*}
\end{theorem}
