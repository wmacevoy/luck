\chapter*{Computation}

Unfortunately, the explicit computations for luck may be impractical.  However, if $n$ independent samples are taken $S=(x_1, \ldots, x_n)$, then it can be estimated as: 
\begin{align*}
 \ell(x_0) = & \frac{1}{n} \left\{\text{\# of outcomes in S more probable than $x$}\right\}  \\
 & + \frac{1}{2n} \left\{\text{\# of outcomes in S equally probable to $x$}\right\} 
\end{align*}

It is a reasonably straightforward calculation to show that
\begin{equation}
E(\ell)=L \,,
\end{equation}
and
\begin{equation}
E((\ell-L)^2) \leq \frac{1}{n} L \cdot (1-L)\,.
\end{equation}
