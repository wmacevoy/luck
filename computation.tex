\chapter*{Computation}

Unfortunately, the explicit computations for luck may be impractical.  However, if $n$ independent samples are taken $S=(x_1, \ldots, x_n)$, then it can be estimated as: 
\begin{align*}
 \ell(x) = & \frac{1}{n} \left\{\text{\# of outcomes in S more probable than $x$}\right\}  \\
 & + \frac{1}{2n} \left\{\text{\# of outcomes in S equally probable to $x$}\right\} 
\end{align*}

It is a reasonably straightforward calculation to show that
\begin{equation}
E(\ell(x))=L(x) \,,
\end{equation}
and
\begin{equation}
E((\ell(x)-L(x))^2) \leq \frac{1}{n} L(x) \cdot (1-L(x))\,.
\end{equation}

\begin{example}{Multinomial.} We are aware of no computationally efficient way to exactly compute the luck for a multinomial distribution other than the explicit sum.  Suppose there is a questionaire with 4 possible answers with category probabilities $0.1$, $0.2$, $0.3$, and $0.4$.  How lucky would it be to get $13$, $15$, $27$ and $45$ responses of each respective answer in 100 samples? 

% This example is small enough to compute the exact luck of $0.6287498$,  indicating this is a fairly typical result.

The log-probability in this case is the multinomial probability (see figure~\ref{fig:mulprobln}):
\begin{equation}
y=\text{mulprobln(x,p)}=\log \left[\left(\sum_{i=1}^{n_p} x_i \right)! \prod_{i=1}^{n_p} \frac{p_i^{x_i}}{x_i!} \right]\,.
\end{equation}

\begin{figure}
\caption{\label{fig:mulprobln}Scilab listing to compute the natural log of the multinomial probability $y$ of outcome $x$ given category probabilities $p$.  $p$ is a $n_{\text{probs}} \times 1$ column vector of category probabilities, $x$ is a $n_{\text{probs}} \times n_{\text{samps}}$ of outcomes, and the result $y$ is a $1 \times n_{\text{samps}}$ row vector of the natural log of multinomial probabilites for these outcomes.}
\lstset{language=Scilab}
\begin{lstlisting}
function y=mulprobln(x,p)
  [nprobs,nsamps]=size(x);
  pln=log(p);
  y=zeros(1,nsamps);
  for i=1:nsamps
    xi=x(:,i);
    ni=sum(xi);
    y(i)=gammaln(ni+1)+sum(xi.*pln-gammaln(xi+1));
  end
endfunction
\end{lstlisting}
\end{figure}

Numerical estimates of luck requires samples from the given probability distribution.  Figure~\ref{fig:mulsamps} uses a built-in function in scilab to generate such a sample set.
\begin{figure}
\caption{\label{fig:mulsamps}Scilab listing to get a sample of outcomes from a multinomial distribution.  $n_{\text{samps}}$ is the number of desired samples, $n_{\text{trials}}$ is the number of trials in each sample, and $p$ is a $n_{\text{probs}} \times 1$ column vector of category probabilities.  The result $x$ is a $n_{\text{probs}} \times n_{\text{samps}}$ matrix of sample outcomes, where $\sum_{j=1}^{n_{\text{probs}}} x(j,i)=n_{\text{trials}}$.}
\lstset{language=Scilab}
\begin{lstlisting}
function x=mulsamp(nsamps,ntrials,p)
  x=grand(nsamps,"mul",ntrials,p(1:(length(p)-1)));
endfunction
\end{lstlisting}
\end{figure}

For smaller spaces, luck for a multinomial distribution can be computed explicitly.  Figure~\ref{fig:mulluck} computes this as an inefficient testing reference.
\begin{figure}
\caption{\label{fig:mulluck}Recursively compute luck of multinomial exactly using exaustive sum.  $x$ is a $n_{\text{probs}} \times n_{\text{samps}}$ matrix of outcomes, and $p$ is a $n_{\text{probs}} \times 1$ column vector of category probabilities.  The result is a $1 \times n_{\text{samps}}$ of luck values.}
\lstset{language=Scilab}
\begin{lstlisting}
function el=mulluckrec(nb,mb,lpx,k,p,y,eps)
  [n,m]=size(lpx);
  el=zeros(n,m);
  for yk=0:nb-mb
     y(k)=yk;
     if k < length(p)-1 then
       el=el+mulluckrec(nb,mb+yk,lpx,k+1,p,y,eps);
     else
       y(length(p))=nb-mb-yk;
       lpy=mulprobln(y,p);
       c=0.5*bool2s(lpy > lpx-eps)+0.5*bool2s(lpy > lpx+eps);
       el = el + c .* exp(lpy);
     end
  end
endfunction

function lucks=mulluck(x,p,eps)
  if ~exists("eps","local") then
    eps=sqrt(%eps);
  end
  [nprobs,nsamps]=size(x);
  ntrials=sum(x,'r');
  min_ntrials=min(ntrials);
  max_ntrials=max(ntrials);
  assert_checkequal(min_ntrials,max_ntrials);
  lucks=mulluckrec(min_ntrials,0,mulprobln(x,p),1,p,zeros(nprobs,1),eps)
endfunction
\end{lstlisting}
\end{figure}

\begin{figure}
\caption{\label{fig:numlucksetup}Given the log of the probabilites of a set of sample data, return a table used for quickly estimating luck.  {\tt problns} is a $1 \times n_{\text{nsamps}}$ row vector of logs of probabilities, and {\tt eps} is an optional parameter giving the absolute error (in log space) for considering two probabilities to be equal.  Returns {\tt setup}, a $3 \times N$ matrix giving estimates for $(L(x),|\Omega(x)|,|\omega(x)|)$ for each unique probability in the sample.  This function is useful generally (not just for multinomial distributions).}
\lstset{language=Scilab}
\begin{lstlisting}
function setup=numlucksetup(problns,eps)
  if ~exists("eps","local") then
    eps=sqrt(%eps);
  end

  n=length(problns);
  problns=gsort(problns);

  i=1;
  count=0;
  while i<=n
    j=i;
    while (j <= n-1 & abs(problns(j+1)-problns(i)) < eps)  
      j=j+1;
    end
    i=j+1;
    count=count+1;
  end

  setup=zeros(3,count);

  i=1;
  count=0;
  while i<=n
    j=i;
    while (j <= n-1 & abs(problns(j+1)-problns(i)) < eps)  
      j=j+1;
    end

    count=count+1;
    setup(1,count)=0.5*(problns(i)+problns(j));
    setup(2,count)=(i-1)/n;
    setup(3,count)=(j-i+1)/n;
    i=j+1;
  end
endfunction
\end{lstlisting}
\end{figure}

\begin{figure}
\caption{\label{fig:numluck}Estimate luck given log of probabilities and setup.  {\tt problns} is a row vector of log-probabilities, {\tt setup} is the setup from a (possibly different) sample, and {\tt eps} is the same optional parameter as in {\tt numlucksetup}.}
\lstset{language=Scilab}
\begin{lstlisting}
function lucks=numluck(problns,setup,eps)
  if ~exists("eps","local") then
    eps=sqrt(%eps);
  end

  [n,m]=size(problns);
  [three,nsetup]=size(setup);

  abs_Omega=zeros(n,m);
  abs_omega=zeros(n,m);

  for i=1:n
    for j=1:m
      lo=1;
      hi=nsetup;
      while %T
        mid=floor((hi+lo)/2)
        if setup(1,mid) > problns(i,j)+eps then
   	  if mid == lo then
            break
          end
          lo=mid;
        else
          if mid == hi then
            break
          end
          hi=mid;
        end
      end

      if abs(problns(i,j)-setup(1,hi))<eps then
        abs_Omega(i,j)=setup(2,hi);
        abs_omega(i,j)=setup(3,hi);
      elseif abs(problns(i,j)-setup(1,lo))<eps then
        abs_Omega(i,j)=setup(2,lo);
        abs_omega(i,j)=setup(3,lo);
      else
        abs_Omega(i,j)=setup(2,lo);
        abs_omega(i,j)=0.5*exp(problns(i,j));
      end
    end
  end
  lucks=abs_Omega+0.5*abs_omega;
endfunction
\end{lstlisting}
\end{figure}

\begin{figure}
\caption{\label{fig:numluckprog}Use the above functions for estimating luck numerically ({\tt nlucks}) and estimating the error in luck (numerical standard deviation).  The last lines optionally compute the exact values of luck ({\tt lucks}) and standard deviations to compare with.}
\lstset{language=Scilab}
\begin{lstlisting}
// get sample set
nsamps=10000;
ntrials=100;
p=[0.1;0.2;0.3;0.4];
x=mulsamp(nsamps,ntrials,p);
problns=mulprobln(x,p);

// setup for luck estimates
setup=numlucksetup(problns);

// estimate luck numerically
x0=[13; 15; 27; 45];
problns0=mulprobln(x0,p);
nluck=numluck(problns0,setup);
nsd=sqrt(nluck .* (1-nluck) ./ nsamps);

// optional exact luck
luck=mulluck(x0,p);
sd=sqrt(luck .* (1-luck) ./ nsamps);

// z is approximately normally disributed
z=(nluck-luck) ./ sd;
\end{lstlisting}
\end{figure}
\end{example}
